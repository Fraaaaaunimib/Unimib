\begin{document}
\documentclass[a4paper,11pt]{article}
\usepackage[utf8]{inputenc} % lettere accentate da tastiera
\usepackage[italian]{babel} % lingua del documento
\usepackage{geometry}
\usepackage{amssymb}
\usepackage{graphicx}
\usepackage{amsmath}
\usepackage{mathrsfs}
\usepackage{booktabs}
\usepackage{fancyhdr}
\usepackage{dsfont}
\usepackage{enumitem}
\usepackage{float}
\usepackage{eurosym}
\usepackage{yhmath}
\usepackage{physics}
\geometry{a4paper,top=2cm,bottom=3cm,left=2cm,right=2.5cm,%
             heightrounded,bindingoffset=5mm}
\newtheorem{theorem}{Th.}
 
\title{\vspace{-2cm}
             \large{\textsc{LICEO SCIENTIFICO \ A.\ VOLTA -- A.S. 2022/2023}}\\[2mm]
             \normalsize \textsc{Verifica di matematica: Funzioni: continuità e derivabilità. Studio.}\\[2mm]
}
\author{}
\date{}
\begin{document}
            
             \maketitle
             \vspace{-2cm}
             \thispagestyle{empty}
            
             \noindent  Classe 5 Sez.... -  Nome:.................................................................... Data: ..../..../.....
             \section*{Quesiti}
 
             \begin{enumerate}
                          
                           \item Calcola le derivate delle seguenti funzioni:
                           \begin{enumerate}
                                        \item \begin{equation*}
                                                      f\left(x\right)=\ln \frac{\sqrt[3]{x}+x}{\sqrt{x}-x}
                                        \end{equation*}
                                        \item \begin{equation*}
                                                     f\left(x\right)=\sqrt{\frac{\left(1-\sin x \right)^2}{\left(1 + \sin x \right)^3}}
                                        \end{equation*}
                           \end{enumerate}          
                          
                           \item Studia gli eventuali punti di singolarità e/o di non derivabilità della funzione:
                                        \begin{equation*}
                                                     f\left(x\right)=\sqrt{\left|\frac{x-2}{x}\right|-\ln \left|\frac{x-2}{x}\right|}
                                        \end{equation*}
                          
                           \item Data la funzione:
                                        \begin{equation*}
                                                     f\left(x\right)=\frac{ax^2+bx-1}{x-c}\qquad \mbox{con}\;\;a,b,c \in \mathbb{R}
                                        \end{equation*}
                           determina i valori di $a$, $b$, $c$ affinché il grafico di $f$ passi per il punto $A\left(0; 1\right)$, abbia come tangente una retta parallela alla retta di equazione $x-2y+8=0$ e per asintoto obliquo una retta parallela alla retta di equazione $4x-y=0$.
                          
                           Dopo aver verificato che le condizioni sono verificate per i valori $a=4$, $b=\frac{1}{2}$, $c=1$, traccia il grafico probabile della funzione.
                          
             \end{enumerate}
            
             \section*{Problema}
                           Considera la funzione:
                           \begin{equation*}
                                        f\left(x\right)=e^{\sqrt{\left|\frac{1-x}{x+2}\right|}}
                           \end{equation*}
            
                           \begin{enumerate}
                                        \item Studia la funzione e tracciane il grafico
                                        \item Pur non studiando la derivata seconda, puoi determinare qualitativamente se la funzione $f$ ammette flessi, il loro numero e la loro posizione? Spiega le tue considerazioni.
                                        \item Determina l'equazione della retta tangente al grafico della funzione $f$ nel suo punto di ascissa $x=0$ e determina le coordinate dei punti di intersezione della tangente con il grafico della funzione $f$.
                           \end{enumerate}
                           \vspace{,5 cm}
\begin{tabular}{lp{0.05\textwidth}p{0.05\textwidth}p{0.05\textwidth}p{0.05\textwidth}p{0.05\textwidth}l}
                           Quesiti             & 1 &  2  &  3  &  &    \\
                           \toprule
                           pti. max            & 1 & 1,5 & 1,5 &  &  4 \\
                           \toprule
                           pti. ass.                            &   &     &     &  &    \\
                           \toprule
                           Problema            &   &     &     &  &  4 \\
                           \toprule
                           pti. ass.           &   &     &     &  &    \\
                           \toprule
                           pti. tot.                            &   &     &     &  &    \\
 
                           punti               &   &     &     &  & ...\\
                           \midrule
             \end{tabular}
 
             \vspace{0,5cm}
             voto (punteggio + 2) =
             \vspace{1cm}
                          
\end{document}